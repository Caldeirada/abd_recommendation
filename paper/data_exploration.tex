\section{Data Exploration}
\label{data_exploration}

Before implementing the algorithm a data analysis is required to better understand the data at hand.

To start the null values were counted to better evaluate the dataset, the result was:
\begin{Verbatim}[commandchars=\\\{\}]
+--------------+--------+-------------+
|category\_code |brand   |user\_session  |
+--------------+--------+-------------+
|      35413780|15341158|           12|
+--------------+--------+-------------+
\end{Verbatim}
The other columns presented no missing values.


The most purchased and viewed brands were also counted, only showing the top 10.
\begin{Verbatim}[commandchars=\\\{\}]
+--------+------+--------+--------+    
|  Purchased    +     Viewed      |
+--------+------+--------+--------+
|   brand| count|   brand|   count|
+--------+------+--------+--------+
| samsung|372923| samsung|11898628|
|   apple|308937|   apple| 9374247|
|  xiaomi|124908|  xiaomi| 7232401|
|  huawei| 47204|  huawei| 2358235|
|cordiant| 27534| lucente| 1775749|
| lucente| 26137|      lg| 1574848|
|    oppo| 25971|   bosch| 1480771|
|      lg| 21606|    oppo| 1203440|
|    sony| 17038|    sony| 1193071|
|   artel| 15391|    acer| 1084065|
+--------+------+--------+--------+
\end{Verbatim}

The most purchased products with the respective brand and number of purchases
\begin{Verbatim}[commandchars=\\\{\}]
+----------+-------+-----+
|product_id|  brand|count|
+----------+-------+-----+
|   1004856|samsung|61265|
|   1004767|samsung|44419|
|   1005115|  apple|34787|
|   4804056|  apple|30181|
|   1004833|samsung|26183|
|   1002544|  apple|22227|
|   1004870|samsung|21288|
|   1004249|  apple|17971|
|   1005105|  apple|15776|
|   1004836|samsung|15549|
+----------+-------+-----+
\end{Verbatim}
The product id is given since the dataset didn't provide the product commercial name.

%We also plotted the information we deemed useful. The results can be seen below
