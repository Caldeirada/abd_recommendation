\section{Dataset}
\label{dataset}

For a recommendation system, the dataset must follow some criteria, such as an evaluation system, where it is possible to create a relationship between users and items.


For this project, we chose a dataset collected by Open CDP from an eCommerce store with more than 15 million visitors per month.
All sensitive and personal data was removed. 
Broken data (like NULL price or products without categories) was removed also.

The file contains behavior data for 2 months (from October 2019 to November 2019) from a large multi-category online store.
Each row in the file represents an event. 
All events are related to products and users.
Each events is like many-to-many relation products and users.

The dataset is structured with 9 columns:
\begin{center}
\begin{tabular}{ | p{3cm} | p{2cm} | p{5cm} | } 
 \hline
 \textbf{Property} & \textbf{Type} & \textbf{Description} \\ 
 \hline
 \textbf{event\_time} & Date & Time when event happened at (in UTC)s. \\
 \hline
 \textbf{event\_type} & String & Only one kind of event. \\
 \hline
 \textbf{product\_id} & Number & ID of a product. \\
 \hline
 \textbf{category\_id} & Number & Product's category ID. \\
 \hline
 \textbf{category\_code} & String & Product's category taxonomy (code name) if it was possible to make it. Usually present for meaningful categories and skipped for different kinds of accessories. \\
 \hline
 \textbf{brand} & String & Downcased string of brand name. Can be missed. \\
 \hline
 \textbf{price} & Number & Float price of a product. Present. \\
 \hline
 \textbf{user\_id} & Number & Permanent user ID. \\
 \hline
 \textbf{user\_section} & String & Temporary user's session ID. Same for each user's session. Is changed every time user come back to online store from a long pause. A session can have a multiple events. \\
 \hline
\end{tabular}
\end{center}

Event types can be:
\begin{itemize}
    \item \textbf{view} - a user viewed a product.
    \item \textbf{cart} - a user added a product to shopping cart.
    \item \textbf{remove\_from\_cart} - a user removed a product from shopping cart.
    \item \textbf{purchase} - a user purchased a product.
\end{itemize}
