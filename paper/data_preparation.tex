\section{Data Preparation}
\label{data_prep}

To better handle the algorithm input data some changes had to be done to the dataset,
some columns could be removed and a new one added.

First, three columns were excluded:
\begin{itemize}
    \item \textbf{event\_time} - The event timestamp was removed because it isn't a relevant data for the algorithm used.
    \item \textbf{category\_code} - This column has null values and the dataset already provides a category\_id column with a numeric value to represent the category
    In order not to have redundant information, we decided to exclude it from our dataset.
    \item \textbf{price} - This data is not relevant for our recommendation system, since the algorithm doesn't take the price into consideration.
    The main focus is to establish a relationship between users and products / categories / brands.
\end{itemize}

The new column, named brand\_category, was the concatenation of two others, the category id and brand. This was done to better 
understand the user preference when purchasing items of a given category and their brand of choice.
An example below

\begin{Verbatim}[commandchars=\\\{\}]
    +-------------------+--------+------------------------+
    |category\_id        |brand   |brand\_category           |
    +-------------------+--------+------------------------+
    |2053013552326770905|aqua    |2053013552326770905.aqua|
    +-------------------+--------+------------------------+
\end{Verbatim}

For this project we decided to just consider the events of \textit{view} and \textit{purchase} type since 
in this dataset the \textit{cart} event type doesn't allow to distinguish between the items added to the cart and the removed ones.
Then the dataset was split by event types mencioned above. After to prepare the data for the algorithm several tables were created by grouping the dataset by user\_id and session\_id and
joining the product\_id, category\_id, brand and the new column brand\_category.
This yielded 14 tables that served as an input for the algorithm.
The tables ending with \textit{\_v} represent the views and the others the purchases.
Each table was then saved into a file.
\begin{itemize}
    \item \textbf{user\_products} - user\_id $\times$ product\_id
    \item \textbf{user\_products\_v} - user\_id $\times$ product\_id
    \item \textbf{user\_brands} - user\_id $\times$ brand
    \item \textbf{user\_brands\_v} - user\_id $\times$ brand
    \item \textbf{user\_categories} - user\_id $\times$ category\_id
    \item \textbf{user\_categories\_v} - user\_id $\times$ category\_id
    \item \textbf{user\_category\_brand} - user\_id $\times$ brand\_category
    \item \textbf{user\_category\_brand\_v} - user\_id $\times$ brand\_category
    \item \textbf{session\_products} - session\_id $\times$ brand\_category
    \item \textbf{session\_products\_v} - session\_id $\times$ brand\_category
    \item \textbf{session\_brands }- session\_id $\times$ brand\_category
    \item \textbf{session\_brands\_v} - session\_id $\times$ brand\_category
    \item \textbf{session\_categories} - session\_id $\times$ brand\_category
    \item \textbf{session\_categories\_v} - session\_id $\times$ brand\_category
    \item \textbf{session\_category\_brand} - session\_id $\times$ brand\_category
    \item \textbf{session\_category\_brand\_v} - session\_id $\times$ brand\_category
\end{itemize}