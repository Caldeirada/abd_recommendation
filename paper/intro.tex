\section{Introduction}
\label{intro}

A recommendation system combines several computational techniques to select personalized items based on the interests of users and conforms to the context in which they are inserted.
Items can be any product, such as movies, books, music, videos, electronic devices, advertisements, etc.
Companies like Google, Youtube, HBO are recognized for their intensive use of recommendation systems.
This system offers these companies a huge competitive advantage.

Recommendation systems arose in response to people's difficulty in choosing a wide variety of products and services.
These systems have been evolving and nowadays they present a complexity and capacity to make recommendations better than the human recommendations themselves.

In order to improve the performance of the recommendations, the system approach must be directed to the relationships between the customers and the items. 
It makes predictions (filtering) about the interests of users and collects the preferences of the various users (collaboration).
For example, if user A has the same opinion as user B about one product, they are more likely to have the same opinion about another product than with a random user.
The similarity between two items is determined by the interest / rating given by the users.

\subsection{Filtering}

Collaborative filtering is a technique used to make automatic predictions about the user's interest by collecting interests and comparing them with other users with the same interests.
A \textbf{Neighborhood methods} is a technic focused on the relationship between items and users.
This approach assesses users' preferences according to the assessment of items in neighboring items given by the same users.
A neighboring item is one that has a similar rating given by the same user.

TODO