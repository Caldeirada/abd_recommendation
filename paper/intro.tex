\section{Introduction}
\label{intro}
In this article it will be explained the structure of the dataset used and the approaches that
were implemented to create the recommendation system.
A recommendation system combines several computational techniques to select personalized items based on the
interests of users and conforms to the context in which they are inserted.
Items can be any product, such as movies, books, music, videos, electronic devices, advertisements, etc.
Companies like Google, Youtube, HBO are recognized for their intensive use of recommendation systems to 
keep customers using their products, either by suggesting  purchases to make or content to use.
This kind of systems gives these companies a huge competitive advantage.

Recommendation systems arose in response to people's difficulty in choosing a wide variety of products and services.
These systems have been evolving and nowadays they present a complexity and capacity to make recommendations better
than the human recommendations themselves. Recommendation systems ingest ever increasingly large datasets in 
order to make the assumptions, and thus posing a difficulty of the analysis.

Several approaches to recommendation algorithms exist, from predicting user ratings to predicting the probality
of different sets of occurring together.

We opted for this last case because it applies better for our dataset since we don't have ratings. 

%In order to improve the performance of the recommendations, the system approach must be directed to the relationships
%between the customers and the items. 

A recommendation algorithm will try to predict a given user response to a certain item(a news article, a movie seen, ...)
based of the history of user, after predicting the rating the ones rated higher are expected to be of the like
of the user.

It makes predictions (filtering) about the interests of users and collects the preferences of the various users (collaboration).
For example, if user A has the same opinion as user B about one product then it is more likely that they share
the same opinion in other products than user A sharing the same opinion with a random user.
The similarity between two items is determined by the interest / rating given by the users.